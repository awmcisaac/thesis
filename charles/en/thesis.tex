%%% The main file. It contains definitions of basic parameters and includes all other parts.
%! TeX program = pdflatex

%% Settings for single-side (simplex) printing
% Margins: left 40mm, right 25mm, top and bottom 25mm
% (but beware, LaTeX adds 1in implicitly)
\documentclass[12pt,a4paper]{report}
\setlength\textwidth{145mm}
\setlength\textheight{247mm}
\setlength\oddsidemargin{15mm}
\setlength\evensidemargin{15mm}
\setlength\topmargin{0mm}
\setlength\headsep{0mm}
\setlength\headheight{0mm}
% \openright makes the following text appear on a right-hand page
\let\openright=\clearpage

%% Settings for two-sided (duplex) printing
% \documentclass[12pt,a4paper,twoside,openright]{report}
% \setlength\textwidth{145mm}
% \setlength\textheight{247mm}
% \setlength\oddsidemargin{14.2mm}
% \setlength\evensidemargin{0mm}
% \setlength\topmargin{0mm}
% \setlength\headsep{0mm}
% \setlength\headheight{0mm}
% \let\openright=\cleardoublepage

%% Generate PDF/A-2u
\usepackage[a-2u]{pdfx}

%% Character encoding: usually latin2, cp1250 or utf8:
\usepackage[utf8]{inputenc}

%% Prefer Latin Modern fonts
\usepackage{lmodern}

%% Further useful packages (included in most LaTeX distributions)
\usepackage{amsmath}        % extensions for typesetting of math
\usepackage{amsfonts}       % math fonts
\usepackage{amsthm}         % theorems, definitions, etc.
\usepackage{bbding}         % various symbols (squares, asterisks, scissors, ...)
\usepackage{bm}             % boldface symbols (\bm)
\usepackage{graphicx}       % embedding of pictures
\graphicspath{{../../figures/}}
\usepackage{fancyvrb}       % improved verbatim environment
\usepackage{natbib}         % citation style AUTHOR (YEAR), or AUTHOR [NUMBER]
\usepackage[nottoc]{tocbibind} % makes sure that bibliography and the lists
			    % of figures/tables are included in the table
			    % of contents
\usepackage{dcolumn}        % improved alignment of table columns
\usepackage{booktabs}       % improved horizontal lines in tables
\usepackage{paralist}       % improved enumerate and itemize
\usepackage{xcolor}         % typesetting in color
\usepackage{nameref}		% reference labels by name
\usepackage{caption}
\usepackage{subcaption}
\usepackage[capitalise]{cleveref}
\usepackage{lscape}
\usepackage[acronym,toc]{glossaries} % for list of abbreviations
\glsdisablehyper
\usepackage{geometry}

% Ondrej's macros for commenting
\usepackage[normalem]{ulem} % sout, uline
\def\XXX#1{\textcolor{red}{XXX #1}}
\def\repl#1#2{\textcolor{red}{XXX \sout{#1}}\textcolor{blue}{\uline{#2}}}


%%% Basic information on the thesis

% Thesis title in English (exactly as in the formal assignment)
\def\ThesisTitle{Temporal Reasoning in Vision and Language Models}

% Author of the thesis
\def\ThesisAuthor{Andrew McIsaac}

% Year when the thesis is submitted
\def\YearSubmitted{2024}

% Name of the department or institute, where the work was officially assigned
% (according to the Organizational Structure of MFF UK in English,
% or a full name of a department outside MFF)
\def\Department{Institute of Formal and Applied Linguistics}

% Is it a department (katedra), or an institute (ústav)?
\def\DeptType{Institute}

% Thesis supervisor: name, surname and titles
\def\Supervisor{doc. RNDr. Ond\v{r}ej Bojar, Ph.D.\\Prof. Raffaella Bernardi\\Prof. Paolo Rota}

% Supervisor's department (again according to Organizational structure of MFF)
\def\SupervisorsDepartment{Institute of Formal and Applied Linguistics}

% Study programme and specialization
\def\StudyProgramme{Computer Science}
\def\StudyBranch{Language Technologies and Computational Linguistics}

% An optional dedication: you can thank whomever you wish (your supervisor,
% consultant, a person who lent the software, etc.)
\def\Dedication{%
Thank you first, to my supervisors at the University of Trento, Prof. Raffaella
Bernardi and Prof. Paolo Rota, for introducing this engaging topic to me, and
helping to guide me through this thesis-writing process, and for providing me
access to compute clusters, without which I would not have been able to do this
work. Thank you also to my co-supervisor at Charles University, Doc. RNDr.
Ond\v{r}ej Bojar, and to all my professors who have taught me throughout this
program. I would also like to thank the Erasmus Mundus Language and
Communication Technologies program for generously providing me with a
scholarship during my Master's, and for giving me the opportunity to travel and
spend an extended period of time in two wonderful countries. 

But the places wouldn't be anything without the people you meet, so thank you
to the friends I have made along the way, for your support and kindness
throughout. Finally, thank you to my family, who have supported me from afar
over the last two years.
}

% Abstract (recommended length around 80-200 words; this is not a copy of your thesis assignment!)
\def\Abstract{%
Are vision and language models able to reason across time? We evaluate the
performance of vision and language models (VLMs) on the task of video question
answering, with a particular focus on their temporal reasoning abilities. We
probe the STAR video QA dataset on two VLMs with data perturbation methods of
text and video inputs, and find that models are generally unable to identify
the meaning of before and after in sequential questions. We then ask how a
model can effectively learn these temporal relations, and design a new dataset
drawn from videos and annotations from the Charades dataset. We create
annotations that include targeted hard negative examples for the contrastive
loss objective of one VLM, Merlot Reserve, such that the model must adapt to
learn temporal relations. We further explore how to model fine-grained temporal
relationships, and evaluate the benefits. We find that our approach shows
promising signs of improvement on tasks that require temporal understanding,
although it gains little sensitivity to temporal relations when probed.
}

% 3 to 5 keywords (recommended), each enclosed in curly braces
\def\Keywords{%
	{multimodal} {video LM} {temporal reasoning} {contrastive learning}
}

%% The hyperref package for clickable links in PDF and also for storing
%% metadata to PDF (including the table of contents).
%% Most settings are pre-set by the pdfx package.
\hypersetup{unicode}
\hypersetup{breaklinks=true}

% Definitions of macros (see description inside)
\include{macros}

% List of abbreviations
%! TeX root = charles/en/thesis.tex

\makeglossaries

\newacronym{llm}{LLM}{large language model}
\newacronym{vqa}{VQA}{visual question answering}
\newacronym{vidqa}{video QA}{video question answering}
\newacronym{clip}{CLIP}{contrastive language-image pre-training}
\newacronym{cnn}{CNN}{convolutional neural network}
\newacronym{nlp}{NLP}{natural language processing}
\newacronym{rnn}{RNN}{recurrent neural network}
\newacronym{lstm}{LSTM}{long short-term memory network}
\newacronym{vlm}{VLM}{vision and language model}
\newacronym{vidlm}{vidLM}{video language model}


% Title page and various mandatory informational pages
\begin{document}
\include{title}

%%% A page with automatically generated table of contents of the master thesis

\tableofcontents

%%% Each chapter is kept in a separate file
\chapter*{Introduction}
\addcontentsline{toc}{chapter}{Introduction}
%! TeX root = ../charles/en/thesis.tex

A bit on how vision and language models perform well on many benchmarks. Then
talk about videos. How vision and language models have been extended to video
tasks via extra training on paired video-caption datasets, frozen LMs.

Introduce the problem. Main question: do they learn temporal reasoning? Are
models able to learn the difference between action X happening before action Y,
versus action X happening after action Y.

What do we do? - Discover that the answer is no, at least for Merlot Reserve.
How? Masking temporal indicator in STAR, no idea if it is before/after.
\XXX{Ondrej: A question for you or Raffa: Discovering that the answer is no is actually fortunate, the easier option. What is unclear is how would we convince ourselves that the answer is yes. What methodology would we use?}

Perhaps this should be explored for some other models as well (e.g. ClipBERT
\cite{lei2021clipbert}, Flamingo \cite{alayrac2022flamingo}, Frozen CLIP
\cite{lin2022evl}, VidIL \cite{wang2022vidil}, Socratic Models
\cite{zeng2023socratic}, VideoCLIP \cite{xu2021videoclip})
\XXX{Ondrej: yes! But primarily focus on the promised Merlot Reserve. If you can easily (one push of a button) do more, do as many as you can.}

So, mask temporal indicators in scripts, add negatives for cues.
\XXX{Ondrej: without knowing the details, I can't imagine any such cues yet. Maybe provide already an example in the introduction.}
E.g. before -\textgreater [after, at the same time, while]
Or possibly masking actions, how to generate negatives for this isn't clear.
Would need multiple masks potentially, quite complicated.

Train with either YT-Temporal-1B or Charades. Does it improve performance on
answering questions that require temporal information? If Charades, probably
need to do other datasets as well, e.g. NextQA, Epic Kitchens.

Hopefully the answer is that it performs better.
\XXX{For ``better'' you need a continuous measure (which you will certainly and easily have), and an improvement in this measure. Yet my high-level methodological question remains: What score in this measure would the model need in order to say ``the model does learn temporal reasoning''.}

The rest of this thesis is organised as follows:

\begin{itemize}
	\item \Cref{chap:bg} goes into the background of language models,
		image recognition models, and vision and language models which combine
		the two modalities. We discuss one popular method, Contrastive
		Language-Image Pretraining (CLIP), and one key downstream task, visual
		question answering. We finish the chapter with a broad overview of video
		language models.
	\item \Cref{chap:rel} explores related work on video language models,
		with a particular focus on work that explores the impact of contrastive
		pretraining. We also look at other approaches for instilling temporal
		reasoning in these models.
	\item In \cref{chap:dataset} we look at the main datasets used, STAR and NExT-QA.
	\item \Cref{chap:method} details experiments that show how current
		models perform on temporal reasoning tasks, and describes our approach
		to generating additional hard negatives focussing on temporal words for
		contrastive training.
	\item \cref{chap:results} shows performance of our model on STAR and NExT-QA.
	\item \cref{chap:discussion} discusses the use of contrastive
		pretraining methods in video and language models, and how our method
		affects performance of temporal reasoning systems.
	\item Finally, the \nameref{chap:conclusion} summarises our findings.
\end{itemize}



%! TeX root = ../charles/en/thesis.tex

\chapter{Background}
\label{chap:bg}

In this chapter, we briefly cover progress in obtaining useful representations
of language (\cref{sec:lm}), images (\cref{sec:imrec}), and efforts to combine
these two modalities (\cref{sec:vlm}). We then look at the extension of vision
and language models to videos (\cref{sec:vidlma}), which introduces the extra
complexity of reasoning across sequences of images, and optionally adding a
further modality, audio. Finally, we explore the literature on temporal
reasoning in language and in vision (\cref{sec:tempreason}).
%We finish the chapter with an introduction to and motivation for studying the
%task of video question answering (\cref{sec:vidqa}).

\section{Language Modeling}
\label{sec:lm}

Language modeling is the task of predicting the next word given some number of
previous words. A neural language model~\citep{bengio2003nlm} performs this by
receiving as input to a feedforward neural network a representation of previous
words in a sequence and outputting a probability distribution over possible
words. The probability of a sequence of $T$ words $w_1^T$ is thus the
combined probability of all words given their context:
$$\hat{P}(w_1^T)=\prod_{t=1}^{T}\hat{P}(w_t|w_1^{t-1}).$$
The conditional probability can be approximated by using a fixed context
length $N$, $$\hat{P}(w_t|w_1^{t-1})\approx\hat{P}(w_t|w_{t-N+1}^{t-1}),$$
greatly reducing the computational requirements for longer sequences. This section
summarises common approaches to modelling text sequences.

\subsection{Recurrent Neural Networks}
\label{ssec:rnn}

The \acrfull{rnn} takes individual items from a sequence, one
at a time, and outputs a prediction based on the single unit and a hidden
state.  The hidden state is a recursive unit learnt from previous hidden
states, so that at timestep $t$, the hidden state $h_t$ is a combination of the
previous hidden state $h_{t-1}$ and the current input $x_t$. The hidden state
is therefore a representation of the entire input sequence up to time $t$. This
avoids the problem faced by feedforward neural language models of only
representing a limited context window of size $N$. In theory, an \acrshort{rnn}
can represent an unlimited context.

In practice, \acrshortpl{rnn} struggle to encode long-distance dependencies
well, with the information encoded in hidden states being biased towards more
recent items of the input, and struggling from the vanishing gradient problem,
whereby repeated matrix multiplications for backpropagation through time drive
the gradient to zero. The \acrfull{lstm}~\citep{hochreiter1997lstm} was
proposed to extend the \acrshort{rnn} by modifying the architecture of the
recurrent unit to include three gates: the forget gate, the add gate, and the
output gate. Combined, these three gates keep the context vector, the previous
hidden state, simple by removing information considered no longer useful, add
useful information from the current input, and output information considered
useful for the current hidden state.

\acrshortpl{rnn} are often used in sequence-to-sequence, or encoder-decoder,
set-ups, in which the input sequence is processed by the encoder section,
creating a context vector which is a representation of the entire input
sequence. This is then fed as the initial hidden state of a decoder network to
generate the output. The benefit of this is that the output size is not related
to the input size. For tasks such as machine translation, image captioning, or
open-ended question answering, the ability to generate an answer is critical.

The bottleneck problem is alleviated slightly by the attention
mechanism \citep{bahdanau2015attention}, where an additional context vector is
used by the decoder to dynamically attend to different hidden states of the
encoder based on the current input token in the decoder. This context vector is
created by a weighted sum of encoder hidden states, recomputed at each timestep
during decoding. Attention with \acrshortpl{rnn} improved the state of the art
in machine translation, particularly on sentences with longer input. It has
also been used in \acrfullpl{vlm} to attend to key parts of the image vector for
visual question answering~\citep{yang2016san}.


\subsection{Transformer}
\label{ssec:transformer}

\citet{vaswani2017attention} introduced the Transformer architecture for
sequence tasks, replacing the recurrent nature of the \acrshort{rnn} and its
variants with multi-head self-attention. This allows for parallel computation
since computation at each timestep is independent of all others, greatly
increasing the ability to train on larger and larger data and model sizes.
Self-attention assigns attention scores to each item of the input sequence
itself, regardless of input size, to compute a representation of the sequence.
The Transformer uses stacked layers of self-attention to capture the many ways
that an input sequence can relate to itself. Each self-attention head can learn
to encode different relationships between sequence tokens, and these heads are
combined and linearly projected into the original dimensionality.
\citet{vaswani2017attention} use attention between the encoder and decoder, so
each position in the decoder can attend to all items of the input sequence, and
further use self-attention in both the encoder and decoder. In the decoder, a
modification is made to prevent knowledge of future information being
generated, masking out all values in the input that correspond to future input
connections. Finally, positional encodings are included for each token to keep
some notion of sequence order that would otherwise be lost from the
\acrshort{rnn} architecture. 

The Transformer achieved state of the art performance on machine translation,
and has since been used as the de facto architecture for many sequence tasks,
in both language and vision. It can be trained in an encoder-decoder setting,
or the two parts can be separated to train only the encoder (e.g.
BERT~\citep{devlin2019bert}), or for text generation using only the decoder
with a simple language modelling objective (e.g. GPT-3~\citep{brown2020gpt3}).
Its ability to scale to larger dataset and parameter sizes has led to massive
improvement in zero-shot and few-shot ability on a wide range of downstream
tasks~\citep{hoffmann2022chinchilla}.

\subsection{Masked Language Modeling}
\label{ssec:mlm}

BERT~\citep{devlin2019bert} uses only the encoder layers of the Transformer to
create strong representations of an input sequence. Since it is trivial to
predict the next token in a sequence when provided with the entire context,
BERT, and its descendants such as RoBERTa~\citep{liu2020roberta}, train on a
masked language modeling objective on unlabeled data, where the task is to mask
some percentage of the input tokens at random, and then predict those masked
tokens.

These representations on their own are not especially useful, but once trained,
they provide a great starting point for finetuning to a specific task, where
there may not otherwise be enough data to learn these rich representations of
language. Downstream tasks can include question answering, natural language
inference, or sentence classification. Pre-training then finetuning has become
a common paradigm due to the relative low cost of finetuning once a large model
has been pre-trained. BERT achieved state of the art on eleven \acrshort{nlp}
tasks, all of which were finetuned in less than an hour on a TPU. BERT has
successfully been adapted to vision and language models, as we will discuss in
\cref{sec:vlm}.


\section{Image Recognition}
\label{sec:imrec}

A key part of video and language models is learning representations of frames
in sequence, which involves the classical tasks of object detection, image
segmentation, and image classification. Much like in \acrshort{nlp}, the
standard approach is to pre-train on large image datasets and finetune to a
specific desired task. Convolutional neural
networks~\citep{lecun1989lenet,krizhevsky2012alexnet,he2016resnet} use
convolution kernels, pooling, and optionally batch normalisation, dense or
residual layers to create representations of image features.
AlexNet~\citep{krizhevsky2012alexnet} uses a multi-layer \acrfull{cnn} to
classify images from ImageNet~\citep{deng2009imagenet}, a dataset of over 15
million images from around 22,000 categories, and was the first to show the
scaling power of large datasets and model sizes for producing strong image
features. 

There have been attempts to combine \acrshort{cnn} architectures with
self-attention mechanisms. This may provide more scope for non-local
computation which may be required on tasks such as object detection with large
objects. However, due to the quadratic cost of self-attention in the number of
pixels, naive implementations are infeasible, and approximations struggle to
scale efficiently~\citep{carion2020detr}. The Vision
Transformer~\citep{dosovitskiy2021vit} does away with the \acrshort{cnn} for
image recognition, and uses an adapted version of the Transformer for greater
scalability.


\subsection{Vision Transformer}
\label{ssec:vit}

\citet{dosovitskiy2021vit} introduced the Vision Transformer, which takes the
impressive performance of the Transformer architecture on sequence tasks and
applies it to image tasks. The authors represent an image as a sequence of
patches of an image, with an extra patch embedding added alongside the
positional embedding of the Transformer to maintain the 2-dimensional
information of an image when projected into a linear sequence. The model is
shown in \cref{fig:vit}. The Vision Transformer matched or exceeded state of
the art on many image classification datasets, while being trained for
comparatively less time. As we discuss in \cref{ssec:clip,sec:mreserve}, it has
been used as the visual encoder for multiple multimodal models due to its
scaling ability~\citep{zhai2022scalingvit}.

\begin{figure}[tp]
	\centering
	\includegraphics[width=0.8\textwidth]{vit.png}
	\caption{The Vision Transformer splits an image into patches, embeds them,
		and feeds them into a Transformer encoder. Classification is learned
		via an MLP head following the encoder. Figure reproduced
		from~\citet{dosovitskiy2021vit}}
	\label{fig:vit}
\end{figure}


\subsection{CLIP}
\label{ssec:clip}

\citet{radford2021clip} introduced \acrshort{clip} (Contrastive Language-Image
Pre-training), which uses the Info-NCE loss~\citep{oord2019infonce} to jointly
learn relationships between encodings of text captions and extracted feature
representations of associated images. The Info-NCE loss trains a multimodal
embedding space to maximise the cosine similarity of matching pairs of captions
and images, while minimising the cosine similarity of non-matching pairs in the
batch. The approach is shown in \cref{fig:clip}. The Info-NCE loss is a
symmetric cross-entropy loss, defined
\begin{align*}
	\mathcal{L} = - \sum_{(i,t) \in \mathcal{B}} 
		\left(\log \mathrm{NCE}(z_{i},~z_{t}) + \log \mathrm{NCE}(z_{t},~z_{i}) \right)
,\end{align*}
where NCE is the normalised cross entropy
\begin{align*}
	\mathrm{NCE}(z_{i}, z_{t}) = \frac{\exp(z_{i}\cdot z_{t}^{+})}
	{\sum_{z\in\{z_{t}^{+},z_{t}^{-}\}} \exp(z_{v}\cdot z)}
\end{align*}
for positive caption $z_{t}^{+}$ matched with image $z_{i}$, while $\{z_{t}^{-}\}$ are
the negative captions from the batch $\mathcal{B}$.

Part of this approach is to use a very large batch, so that there are many
incorrect pairings to learn from. \citet{radford2021clip} use a batch size of
32768. The text encoder is a Transformer~\citep{vaswani2017attention}, and
their best model uses a Vision Transformer~\citep{dosovitskiy2021vit} as the
image encoder.

The model enables zero-shot transfer to many downstream computer vision
classification tasks by predicting the most probable (image, text) pair when
given an image and a set of text prompts with each class embedded in the prompt
achieving performance comparable to or surpassing the previous state of the art
by finetuned models. The representations learned by the contrastive
pre-training objective have wide applicability to a range of \acrshortpl{vlm}
and \acrfullpl{vidlm}, particularly as frozen features from which to add
smaller modules on top for adapting to vision and language
tasks~\citep{alayrac2022flamingo,lin2022evl,luo2022clip4clip}. We discuss some
of these models in \cref{sec:vidlma}, and consider the limitations and possible
expansions of the contrastive pre-training method in \cref{sec:contrastive}.

\begin{figure}[tp]
	\centering
	\includegraphics[width=\textwidth]{CLIP.png}
	\caption{CLIP. Given a batch of image-text pairs, the pre-training
	objective matches correct pairs, while minimising similarity of
	non-matching pairs. This representation can be used for a range of
	downstream tasks. From~\citet{radford2021clip}}
	\label{fig:clip}
\end{figure}


\section{Vision and Language Models}
\label{sec:vlm}

One criticism of \acrfullpl{llm} is that the representations learned by
training a model to predict the next word fail to learn any kind of meaning
without reference to the real world~\citep{bender2020climbing}. Models trained
in this way learn connections between surface forms, but no grounded meaning
between the form and intent portrayed through the form. A way to create
grounded representations may be to combine the two modalities of language and
vision through multimodal embeddings. A key challenge in recent years has been
to find suitable methods for creating these shared representations.

One approach follows the strong performance of BERT~\cite{devlin2019bert} in
\acrshort{nlp} tasks. \citet{li2019visualbert} extend BERT to include visual
features extracted from a \acrshort{cnn} as well as text tokens as input to a
Transformer encoder, implicitly discovering a joint representation between the
two modalities. The authors use the self-attention mechanism to align elements
of the input text and regions in the input image, and pre-train on two
visually-grounded language model objectives. The authors finetune and evaluate
on a range of vision and language applications, including visual question
answering (VQA, see~\cref{ssec:vqa}) and visual commonsense reasoning, which
extends \acrshort{vqa} to ask the model to also predict a rationale given a
question and answer pair. This simple method provided encouraging results on
these tasks.

Other works use novel ways of fusing visual information into the language
model. \citet{yu2022coca} use cross-attention layers in a text decoder to train
a captioning loss with pooled features from the image encoder, combined with a
contrastive loss between image and text features. \citet{li2023blip2}
bootstraps a \acrshort{vlm} from separate frozen models for vision and for
language, with a trainable Querying Transformer that extracts a fixed number of
visual features from the vision encoder and optimises to extract features that
most align with the text caption associated with the image. The output of the
Querying Transformer is then trained alongside a frozen \acrshort{llm} to
return interpretable visual features as a prefix to a language model, acting as
an information bottleneck for downstream use by the \acrshort{llm}. One of the
downstream tasks is \acrlong{vqa}.

\subsection{Visual Question Answering}
\label{ssec:vqa}

\Acrfull{vqa} is the task of answering questions given text and an
image. There are a range of datasets that test different aspects of image and
text understanding (e.g.~\cite{johnson2017clevr,hudson2019gqa,antol2015vqa}). A
challenge in creating datasets is to ensure that there are few statistical
biases or shortcuts in the answer distribution that a model can exploit without
a true understanding of the scene. For example, models trained on the VQA
dataset~\citep{antol2015vqa} were found to make predictions based on overly
strong language priors without considering the associated
image~\citep{zhang2016yin} (a green banana may trip up a model) and failed to
show complete question understanding, settling on an answer before receiving
the full question ~\citep{agrawal2016analyzing}. Questions generally required
little reasoning or compositionality, with many answers achievable solely by
object recognition~\citep{hudson2019gqa}. The GQA dataset~\citep{hudson2019gqa}
is one dataset that aims to limit these issues by generating questions with
linguistic diversity and a large vocabulary, and balancing the answer
distribution through sampling. 

Visual question answering has also been extended to question answering over
videos. On top of scene understanding, video question answering requires event
understanding to understand causal and temporal relationships within the
context of the video. A particular challenge in developing models for this task
is combining and aligning the modalities of text, vision, and audio as well.
Several datasets have been proposed for this
task~\citep{xu2016msr-vtt,wu2021star,xiao2021nextqa,lei2020tvqaplus}. We
discuss two which we study and evaluate in our experiments,
STAR~\citep{wu2021star} and NExT-QA~\citep{xiao2021nextqa}, in
\cref{chap:dataset}.

\section{Video Language Models}
\label{sec:vidlma}

%Models which solve video tasks. How to choose frames, methods for learning
%temporal aspect, modeling sequences of images
Much as the task of visual question answering has been extended to the domain
of video, so too have models been created for video language tasks. Video
language models are models used to solve problems related to video
understanding tasks. This introduces the added complexity of temporal modeling
to understand relationships between successive frames in the video, as well as
the possibility of modeling audio where the data allows it. There have been two
main approaches to solving these tasks. The first is to adapt pre-trained vision
and language models as seen in~\cref{sec:vlm} to the new domain without any specific
training or finetuning on a video dataset. This can benefit from the large
amount of research into these models, with huge pre-trained and highly
performant models readily available, although there is a challenge to adapt to
the domain shift and new challenges posed by videos. Alternatively, we can
train on a video dataset, either from scratch, or finetuning from a pre-trained
vision and language model. This section explores both methods.

\subsection{Adapted from Vision and Language Models}
\label{ssec:adaptvlm}

Pre-trained generative \acrfullpl{llm} have shown strong capabilities for
in-context learning~\citep{brown2020gpt3}. In-context learning provides a
number of examples of a task (for few-shot learning -- zero-shot learning
provides only a task description) as the start of a prompt to a language model,
where a typical example contains the context of the task and its desired
completion.  The language model must then provide the correct completion when
presented with just the example context. This idea, and extensions, have been
shown to be effective for a wide range of tasks, particularly those which
require advanced reasoning~\citep{wei2022cot,kojima2022step}.

By providing generative \acrshortpl{llm} with access to image features in its
prompt, it is possible to leverage pre-trained \acrshortpl{llm} for video
tasks.  \citet{wang2022vidil} use a vision and language model to label objects,
events and attributes, as well as captions, for each sampled frame in a video.
These features are then composed in a template for few-shot learning of video
tasks.  Temporal relationships between frames are modeled in the template using
textual indicators (`first', `then', `finally'). Crucially, no finetuning of
language or vision and language models is performed, so high quality
pre-trained models can be plugged in and changed easily. This process is highly
dependent on strong visual feature extraction, meaning that key low-level
features may be lost if the vision models are not strong enough. Concurrent
work by~\citet{zeng2023socratic} finds that using stronger \acrlongpl{vlm}
correlates with better performance when combining \acrshortpl{vlm} and language
models in a zero-shot manner for egocentric perception. The added latency costs
of having separate models for visual tokenisation, frame captioning, and
language generation may however make the overall system inadequate for
practical use.

\citet{portilloquintero2021clipvidret} use \acrshort{clip} features with an
aggregation function across frames to adapt to the video domain for retrieval
tasks. The best aggregation function tested was to simply average frame-level
features, which beat previous best recall@1 scores on the
MSR-VTT~\citep{xu2016msr-vtt} dataset, despite the lack of relative temporal
awareness by mean pooling features from multiple frames. The authors found that
using a single frame from around one second in to the video as the aggregation
function gives significantly worse recall performance than other aggregation
functions which take into account multiple frames.

By contrast, \citet{huang2018videotemporal} found that temporal understanding
plays just a small role in multiple video datasets. On two action recognition
datasets, the impact of motion accounts for just 6 percentage points of 79\%
accuracy on UCF101~\citep{soomro2012ucf101}, and 5 points of 47\% accuracy on
Kinetics~\citep{carreira2018kinetics600}, and 40\% and 35\% of classes do not
require any temporal understanding for the two datasets respectively.
\citet{buch2022revisiting} extend this finding for video language tasks, with
single frame understanding performing strongly compared to state of the art
models, ``even in settings intended for complex multi-frame event
understanding''.  The key distinction between these works
and~\citet{portilloquintero2021clipvidret} is that the model selects a highly
informative frame based on its task. Similarly, \citet{lei2023revealing} find
that training on a randomly chosen single-frame and only providing visual
features from multiple frames at inference time achieves strong performance on
existing datasets, which have a bias towards static appearance.
\citet{buch2022revisiting} propose that their design, the atemporal probe
(ATP), be used to design better datasets that better test efficacy of a
benchmark for causal and temporal understanding. They find a subset of
NExT-QA~\citep{xiao2021nextqa} questions that ``truly necessitate video-level
understanding compared with the original dataset''. We test on both NExT-QA and
this subset, denoted ATP\textsubscript{hard}, in our experiments.


\subsection{Training on Videos}
\label{ssec:vidtrain}

%Models which take into account temporal nature and train/finetune on video datasets

This section mainly explores models pre-trained with a contrastive objective,
since we are predominantly concerned with how models trained with this choice
of objective function, popular for \acrshortpl{vidlm}, are able to learn
temporal reasoning. We note, however, that other
models~\citep{lei2021clipbert,xu2021vlm,alayrac2022flamingo} have achieved
comparable downstream performance when trained with other objectives (masked
language modeling, image-text matching, cross-modal attention).

The obvious approach for training \acrlongpl{vidlm} is to train on videos.
\citet{luo2022clip4clip} extend \acrlongpl{vlm} to video retrieval in a simple
way by mapping sequences of image representations learned from \acrshort{clip}
into a fixed video representation, and computing similarity between the
\acrshort{clip} text encoding and the learned video encoding. They find that
training on a medium-sized video dataset starting from the \acrshort{clip}
encodings improves zero-shot and finetuning results on multiple downstream
datasets, and that attempts to model the temporal dependency between frames
(using 3D linear projections for video features, and using similarity
measurements that model sequentiality for the video and text similarity
measure) from the base \acrshort{clip} model trained only on image and text
pairs do not produce better results on video tasks.

\citet{bain2021frozen} learn a separate visual encoder for images and videos,
and a text encoder for captions for video retrieval. Visual features are used
as input to a space-time Transformer encoder, which, when projected into a
common video-text space, is contrastively compared to the encoded text
features. The authors use curriculum learning to learn the temporal information
of videos by increasing the number of frames provided to the model during
training. They find that training on a single frame is not enough for
retrieval, and that progressively increasing the number of frames (up to 8
frames) during pre-training can result in better performance than training with
more frames to start with. 

\cite{xu2021videoclip} train a video understanding model using a contrastive
objective with loosely temporally aligned videos and text transcriptions. The
authors note that strict alignment between transcript and video clips can
result in low relevance pairings. 
%In the case of an instructional video, the
%domain of the pretraining dataset used~\citep{miech2019howto100m}, a sentence
%such as ``I am going to show you how to cook fried rice'' may not be exactly
%aligned with the visual semantic content relating to the sentence. 
The authors
empirically find that loosening the constraint between video and text clip
timestamps to have loosely overlapping pairs provides a better association
between video and text pairs. The authors also create batches based on
semantically similar video/text embeddings, creating hard negative examples
to strengthen the learned embeddings. The retrieval process is intertwined
with the training process, so that as the joint embedding space is learned
harder batch videos can be retrieved.

%Similar to the second stage of the Querying Transformer in \citet{li2023blip2}
%(see \cref{sec:vlm}), \citet{alayrac2022flamingo} train a Perceiver Resampler
%that learns fixed-size visual feature tokens from a variable number of visual
%features. The Perceiver Resampler explicitly includes temporal positional
%encodings corresponding to frames of a video.  This allows the model, Flamingo,
%to train on a combination of image-text and video-text pair datasets, and
%reduces complexity when training on long videos with many visual features. The
%output is then fused into a frozen \acrshort{llm} with trainable
%cross-attention layers with combined visual and language features, allowing the
%\acrshort{llm} to generate text output conditioned on the visual features from
%the Perceiver Resampler.

Finally, \citet{zellers2022mreserve} uses a contrastive span objective, where
videos, speech and their subtitles are aligned in short time spans, and the
model must predict masked out text and audio spans given a frame and the
surrounding context. It is able to scale to loosely aligned datasets, and
outperforms even some finetuned models in a zero-shot setting on the
STAR~\citep{wu2021star} benchmark. We use this model as the basis for our
experiments, and discuss the full architecture in~\cref{sec:mreserve}.

% maybe add a table here showing results from all these different models on
% a couple of tasks. control for pretraining data, ft/zs, 

\section{Temporal Reasoning}
\label{sec:tempreason}

We finish this chapter with an overview of how temporal reasoning is defined in
video and in language, and datasets used to explore the abilities of models in
this direction.

%\noindent\textbf{In Video.}\hspace{0.2cm}
\subsection{In Video}
\label{ssec:tempvid}
Most computer vision has studied how to model concepts and relationships
between them in the world, e.g. through object detection and segmentation in
images. To go one step further into the video domain, we need to study how to
model event knowledge. That is, how do we model recurring and meaningful
patterns and sequences of behaviour? A model must understand activities and
their components, as well as the temporal ordering of these activities to find
causal dynamics between them~\citep{elman2019event}. New datasets and models
have been proposed in recent years that aim to find computational models
capable of this event knowledge through temporal reasoning. 

As discussed in~\cref{ssec:adaptvlm}, some models can still perform well on
video datasets with just a single frame given to the model. This suggests a
requirement for more challenging datasets and tasks to learn temporal ordering
of events. \citet{grauman2022ego4d} create Ego4D, a dataset of over 3000 hours
of egocentric (first-person) video, with several associated tasks requiring
understanding of how objects change state over time, remembering temporal
windows for objects appearing in scenes, and prediction of future actions in
videos, requiring causal understanding of actions and events. For example, a
cooking video may predict the subsequent steps to making a pizza when presented
with the first steps of rolling and kneading dough. The authors identify
normalised pointwise mutual information as a means to inform the temporal
structure of sequences of actions over time, with certain action sequences
favoured over others. Learning this structure of action pairs is key to a
model's performance on these tasks.

%\noindent\textbf{In Language.}\hspace{0.2cm}
\subsection{In Language}
\label{ssec:templang}
There is a long history of studying temporal expressions in linguistics.
~\citet{moens1988temporal} claim that \textit{when}-clauses ``establish a
temporal focus'' between two events, contingent on e.g. a causal link, as in
the unnatural use of \textit{when} in ``*When my car broke down, the sun set.''
Any representation looking to model temporal descriptions must therefore, for
\textit{when}-clauses and similar phenomena, model contingency, rather than
just temporality.

\citet{allen1983interval} suggests a model of temporal reasoning based on
intervals and relations among them. Given two events, the temporal relations
between them can be expressed in many ways based on the time intervals of the
events occurring. We explore the use of this temporal representation further
in~\cref{sec:data}. \citet{zhou2021tracie} propose a dataset for natural
language inference of temporal relations for before and after relations. They
find that current models struggled to predict temporal relationships between
explicit and implicit events, and that a neuro-symbolic method improved 
reasoning ability by estimating event durations to infer implicit end times.

%Traditional methods
%\cite{bruce1972temporalqa}
%\cite{pustejovsky2003timeml}

%Pre-trained LLMs struggle on temporal reasoning, need finetuning.~\citet{vashishtha2020temporal}

%\noindent\textbf{Probing video datasets.}\hspace{0.2cm}
\subsection{Probing Video Datasets}
\label{ssec:tempprobes}
\citet{sevilla-lara2021temporal} create a perceptual test to discover action
classes in videos that require temporal information to identify. The authors
shuffle frames in time from action classification datasets, and present human
annotators with the shuffled or control videos, where there is no shuffling.
Action classes are then identified by the largest average performance
degradation of action classification between the two groups. They train video
models on a temporal and static dataset, the 50 classes where human accuracy
decreases most and least respectively, and find that training on the temporal
dataset produces features that are more sensitive to temporal ordering, and
therefore are stronger temporal features.
We extend this finding and explore the performance of various models with
frames shuffled on video question answering datasets in \cref{chap:probe}, and
develop a novel method for training \acrlongpl{vidlm} on a temporal-aware
dataset in \cref{chap:setup}.

%\section{Video Question Answering}
%\label{sec:vidqa}
%
%\subsection{STAR Dataset}
%\label{ssec:star}
%
%We primarily focus on this due to its focus on sequential questions that evaluate model performance on temporal reasoning
%
%\subsection{Merlot Reserve}
%\label{ssec:mreserve}
%
%We examine a specific video language model, Merlot Reserve \citep{zellers2022mreserve}, that performs strongly on STAR.

%! TeX root = ../charles/en/thesis.tex
\chapter{Related Work}
\label{chap:rel}

This chapter looks at previous work on probing \acrlongpl{vlm},
and techniques for improving reasoning in various directions in \acrlong{vlm}s. We use
and extend the approaches explored to try and improve the temporal reasoning
abilities of \acrlongpl{vidlm}.

%\section{Video Language Models}
%\label{sec:vidlmb}
%
%CLIP-based \citep{radford2021clip} with no video data, 
%Contrastive with video data, Merlot Reserve~\cite{zellers2022mreserve}, VideoCLIP~\cite{xu2021videoclip}, \cite{luo2022clip4clip}
%Masked VLM: VideoBERT \citep{sun2019videobert}, VLM \citep{xu2021vlm}
%Separate frozen image/video approaches with finetuning/adapters: Flamingo \citep{alayrac2022flamingo}
%Separate frozen image/video approaches with prompt engineering: \citep{wang2022vidil, zeng2023socratic}


\section{Contrastive Training in \acrshortpl{vlm}}
\label{sec:contrastive}

Some previous studies have looked at the effect of contrastive pre-training in
vision and language models, and introduce the idea of post-pretraining VLMs
with hard negatives~\citep{yuksekgonul2023when, momeni2023verbs,
bagad2023testoftime}. Post-pretraining is a continuation of self-supervised
pre-training on a smaller dataset with desired properties that aid the learning
process of the model, mitigating the cost of expensive general pre-training
while allowing for specialisation of a model. This can be used for, e.g.
transferring \acrshortpl{vlm} to the video domain with a small video dataset,
as in~\citet{luo2022clip4clip}, discussed in \cref{ssec:vidtrain}. Or, as we
discuss in this chapter, instilling better understanding of concepts and
relationships with targeted hard negatives in a contrastive objective.
\citet{yuksekgonul2023when} explore compositional relationships in
\acrlongpl{vlm} by testing existing \acrshortpl{vlm} on a dataset with
perturbations exploring attributive understanding of adjectives to nouns,
relational understanding for prepositions and verbs, and sensitivity to word
order in image captions. When presented with an original caption and its
transformation(s), models must predict which caption is more likely. The
authors find that most models are deficient in relational understanding tasks
(e.g. choosing between `the horse is eating the grass' and `the grass is eating
the horse'), but are better at attribution of properties to objects, as in `the
paved road and the white house' vs 'the white road and the paved house'. Models
also performed close to chance on the word order sensitivity test, where
multiple extra captions were created with shuffled nouns/adjectives, shuffled
trigrams, and shuffled words within each trigram, indicating the
\acrshortpl{vlm} behave like bags-of-words.

The authors claim that this may be down to the contrastive pre-training
objective in \acrshortpl{vlm} such as \acrshort{clip}~\citep{radford2021clip},
where the retrieval nature of the objective leads to a bias towards object
recognition without considering compositionality, and that in datasets without
carefully constructed caption alternatives, order information is not required
to solve the objective. An incentive, in the form of additional hard negatives
in both alternative images and generated targeted captions, is therefore
proposed~(\cref{fig:bow_hard_negs}), which improves performance on the testing
benchmarks for attribution (62\% to 71\%), relation (63\% to 81\%), and order
(46\% to 86\% and 59\% to 91\%) substantially, while not degrading performance
in other downstream tasks.

\begin{figure}[t]
	\centering
	\includegraphics[width=0.8\textwidth]{bow_hard_negs.png}
	\caption{Hard negatives for contrastive learning, with generated negative
	captions and retrieved alternative images. Captions are generated by
	swapping various linguistic features, while images are sampled from 
	k-nearest neighbours. From~\citet{yuksekgonul2023when}.}
	\label{fig:bow_hard_negs}
\end{figure}


\section{Understanding in Video Language Models}
\label{sec:understandvidlm}

Here we discuss two papers that look at improving understanding in
\acrfullpl{vidlm} by extending the contrastive objectives with hard negatives
for verb understanding~\cite{momeni2023verbs}, and in before/after
relations~\cite{bagad2023testoftime}. 

\subsection{Verbs in Action}
\label{ssec:verbs}
\citet{momeni2023verbs} look at \acrshortpl{vidlm} trained with a contrastive
loss function, and find similar issues with verb understanding to those
identified by~\cite{yuksekgonul2023when}. They propose to generate hard
negatives with modified verb phrases using pre-trained \acrfullpl{llm}, as well
as introducing an additional verb phrase alignment loss which contrastively
compare a verb phrase from the positive caption to other verb phrases in the
batch to provide an additional focus on verbs to the model
(see~\cref{fig:vfc}). As in~\cite{yuksekgonul2023when}, models trained with
targeted alternatives improve performance for datasets that require
understanding of the targeted domain in both zero-shot and finetuning setups. 

\begin{figure}[t]
	\centering
	\includegraphics[width=\textwidth]{vfc.png}
	\caption{Verb-Focused Contrastive learning. Generated negative captions are
	added as hard negatives to the contrastive loss objective.
	From~\cite{momeni2023verbs}.}
	\label{fig:vfc}
\end{figure}


%InfoNCE loss, work in similar training styles for different domains
%\citep{momeni2023verbs, yuksekgonul2023when}.

\subsection{Test of Time}
\label{ssec:testoftime}
The most similar paper to our work is~\cite{bagad2023testoftime}. The authors
look at before/after relations in videos by using a synthetic dataset to
probe existing models. They construct videos containing pairs of events such
as ``a red circle appears before a yellow circle'', and create distractor
annotations by reversing the order of events but keeping the temporal relation
the same. On this time-order probing task, they find that existing models
perform no better than chance for the task of associating the correct
annotation to the video. They describe a post-pretraining strategy, Temporal
Adaptation by Consistency of Time-order (TACT), for improving the understanding
of before/after relations in \acrshortpl{vidlm}, where non-overlapping video
clips are stitched together and paired with a text description that consists of
the clip captions and a temporal relation, either before or after, to match the
order of the combined video clip (see~\cref{fig:tact}). A reversal function is
then applied to create hard negative examples by reversing the order of text
captions or video clips, which are included as negatives in the contrastive
post-pretraining objective.

\begin{figure}[t]
	\centering
	\includegraphics[width=0.8\textwidth]{tact.png}
	\caption{Temporal Adaptation by Consistency of Time-order. Extra negatives
	are included by reversing time-order in annotations and videos. Figure
	from~\citet{bagad2023testoftime}.}
	\label{fig:tact}
\end{figure}

They find that this approach improves performance on the time-order probing
task, with models much more likely to match the correct caption to the stitched
video clips. On downstream tasks, they find a mixed result. On video retrieval
tasks, on which they claim existing datasets have more of a bias towards
spatial understanding than temporal reasoning
(see~\cref{ssec:adaptvlm};~\citet{buch2022revisiting,lei2023revealing,luo2022clip4clip}),
the model performs slightly worse in general than without using the TACT
approach. On \acrlong{vidqa}, with temporally challenging datasets, there are
generally slight improvements. For example on the NExT-QA
ATP\textsubscript{hard} subset, the TACT model trained on clips from
TEMPO~\citep{hendricks2018tempo} achieves a zero-shot accuracy of 27.6,
compared to 25.0 on the baseline model.  However, on clips from the Charades
dataset~\citep{sigurdsson2016charades} the TACT performance is worse than the
Charades baseline model (25.2 vs 26.0). They find that the TACT model generally
improves performance on action recognition benchmark subsets which have been
identified as requiring temporal information. 

In comparison to~\cite{bagad2023testoftime}, we explore different ways of
probing temporal understanding, use a wider range of temporal relations with
full videos, and aim to gain a stronger relationship between frame and action
with the contrastive span objective. We provide a full comparison of our
approaches and results in~\cref{sec:tactcompare}. We now go on to describe the
datasets and models that we use in our approach.

%! TeX root = ../charles/en/thesis.tex
\chapter{Datasets and Models}
\label{chap:dataset}

This chapter describes the existing datasets and models used in our experiments
in detail.  We first look at two video question answering datasets,
STAR~\citep{wu2021star} and NExT-QA~\citep{xiao2021nextqa}, and then discuss the
model we adapt for use in our experiments, Merlot
Reserve~\citep{zellers2022mreserve}. STAR is tested zero-shot in Merlot Reserve,
and achieves state of the art performance. In \cref{chap:method,chap:results}
we test how we can modify this model to improve its temporal reasoning ability,
while keeping strong performance on the STAR dataset. We further use NExT-QA to
test the generalisability of our model.
%TODO: Use NexT-QA ATP\textsubscript{hard} subset as well.

\section{STAR}

STAR~\citep{wu2021star} is a dataset for situated reasoning in real-world videos.
It uses videos taken from the Charades~\citep{sigurdsson2016charades} dataset,
which describe daily life actions or activities in indoor scenes. A video
is annotated with actions and timestamps. STAR builds a detailed scene annotation
from these videos. A situation is a description of entities, events, movements,
and environments. An example is shown in \cref{fig:star}.

\begin{figure}[htpb]
	\centering
	\includegraphics[width=0.8\textwidth]{star}
	\caption{An example instance from the STAR dataset, with the four question
		types. Image from~\citet{wu2021star}}
	\label{fig:star}
\end{figure}

There are four types of question: interaction, sequence, prediction, and
feasibility.  Based on the type of question, a situation will include complete
action segments or, for prediction and feasibility questions, involve actions
involved in the questions and an incomplete action segment about answers.
Answers are generated to provide three different distractors along with the
correct answer. The compositional distractor satisfies verb-object
compositionality and is generated so as to be feasible in the same situation.
The random option is selected from other instances, with the constraint that
compositionality is satisfied, while the frequent option selects the most
frequently occurring answer in each type of question group to deceive models
that look for shortcuts in this way.

With respect to temporal reasoning, of particular interest to us are the
sequence questions. These are questions which evaluate the temporal reasoning
of systems when facing consecutive actions in dynamic situations, and ask about
relationships between people and objects through the actions they perform in a
situation. The best baseline model achieves an average accuracy of 36.7\%
across all question types. In their baseline results, the authors note that
visual perception has a significant impact on situated reasoning. Existing
vision models struggle to reason well in real-world situations, and models
struggle more to identify relationships between objects than objects
themselves. It is therefore the task of an improved video and language model to
better realise these relationships.

\section{NExT-QA}

%Description. Explanation of temporal questioning. Used for evaluation in different domain
NExT-QA~\citep{xiao2021nextqa} is another video question answering dataset with
a focus on a wider range of temporal actions. Where STAR asks questions that
test only before/after temporal relations, questions in NExT-QA challenge the
model to reason about causal actions as well as temporal relations such as
`when` (\cref{fig:nextqa}).

\begin{figure}[htpb]
	\centering
	\includegraphics[width=0.8\textwidth]{nextqa}
	\caption{Example NExT-QA video and question types.
		From~\citet{xiao2021nextqa}}
	\label{fig:nextqa}
\end{figure}

\section{Merlot Reserve}

Merlot Reserve~\citep{zellers2022mreserve} is a pre-trained video language model
which uses a contrastive objective that learns from aligned audio, subtitles
and video frames. The authors collect a diverse dataset of 1 billion frames
from YouTube videos. Videos are filtered to be high quality, favouring
instructional videos so that there would be a visual grounding to the subtitles
and audio. The architecture consists of independent encodings of each modality,
via a Vision Transformer~\citep{dosovitskiy2021vit}, an Audio Spectrogram
Transformer~\citep{gong2021ast}, and a Transformer span encoder, which computes
targets from an embedding of a candidate text span
(\cref{fig:mreservearch}). This is then fed into a joint Transformer encoder
for all modalities and timesteps.

The specific pre-training objective is called contrastive span training
(\cref{fig:contrastivespan}). For each frame aligned with an image, text,
and audio encoding, a region of text and audio is masked out. The model must
maximise its similarity only to an independent encoding of the text and audio.
In pre-training the model learns to predict spans of text and audio in two
cases: predicting audio where frames and text is provided; and predicting text
where frames and audio are provided. 

\begin{figure}[htpb]
	\centering
	\begin{subfigure}[b]{0.45\textwidth}
		\centering
		\includegraphics[width=\textwidth]{mreservearch}
		\caption{Merlot Reserve Architecture}
		\label{fig:mreservearch}
	\end{subfigure}
	\hfill
	\begin{subfigure}[b]{0.45\textwidth}
		\centering
		\includegraphics[width=\textwidth]{contrastivespan}
		\caption{Contrastive Span Training}
		\label{fig:contrastivespan}
	\end{subfigure}
	\caption{Merlot Reserve Details. Figures taken from~\citet{zellers2022mreserve}}
	\label{fig:mreserve}
\end{figure}

Once pre-trained, the model can be used by finetuning on a dataset, or
zero-shot for a range of video and language tasks. The authors achieved state
of the art results on visual commonsense reasoning~\citep{zellers2019vcr},
TVQA~\citep{antol2015vqa}, another video question answering dataset, and
Kinetics-600~\citep{carreira2018kinetics600}, for activity understanding.
Further, zero-shot experiments showed performance competing with those of
supervised models on a range of video question answering datasets, and even
slightly exceeding the supervised state of the art on STAR.

%! TeX root = ../charles/en/thesis.tex
%\chapter{Method}
%\label{chap:method}

\chapter{Probing Temporal Ability}
\label{chap:probe}

%\XXX{With the current limited writeup, Ondrej cannot comment yet. Some critical remarks: always try hard to complement other people's scores (which you can cite from a paper) with \emph{your own measurement} of their outputs. Typically, these reported and re-measured numbers will differ and it is always important to know them for a reliable comparison with your own scores.}

In this chapter we probe the existing temporal reasoning abilities of two
models: Merlot Reserve~\citep{zellers2022mreserve} and
VideoCLIP~\citep{xu2021videoclip}.  These models are chosen because of their
use of a contrastive pre-training objective and zero-shot ability to test on
downstream datasets. We describe the required adaptation of STAR to allow for
zero-shot testing of both models before detailing results of targeted data
perturbation for temporal features. The perturbations aim to identify how a
model performs with incorrect data. We would expect a model to be more
uncertain and for accuracy to go down.

\section{Zero-Shot Setup}
\label{sec:mres_zs}

As in~\citet{zellers2022mreserve}, we convert questions into statements to more
closely match the text seen by the model during pre-training. Statements are
rewordings of questions with answers masked out, since YouTube captions do not 
typically render question marks. Since not all of the conversion templates used
for the Merlot Reserve model are given, we develop our own based on examples. 
We use STAR's question-answer template program and modify each question type
into a statement type. For example, for the question and answer pair

\texttt{Q: ``What happened after the person put down the towel?''}

\texttt{A: ``Threw the clothes.''}\\
we are given the (slightly modified) question-answer template from STAR

\texttt{Q: ``What happened after the person [VBP] [NP]?''} 

\texttt{A: ``[Answer]''}.\\
The corresponding statement is

\texttt{S: ``The person \_ after they put down the towel.''}

\texttt{A: ``threw the clothes''},\\
or in templated form

\texttt{S: ``The person \_ after they [VBP] [NP].''}

\texttt{A: ``[answer]''}.\\
where the task becomes correctly identifying the correct answer to replace the
masked span denoted by ``\_''. A full conversion from question type to
statement is given in~\cref{tab:qs_to_stmts}.

Given these statements we present the model with the masked statement for each
instance. This is aligned with the first frame, and all other frames have no
aligned text. The correct answer is the choice most likely to replace the
masked token(s).

%\section{VideoCLIP Zero-Shot Setup}
%\label{sec:videoclip_zs}

For VideoCLIP, we use a similar approach to the zero-shot video question
answering approach in~\citet{xu2021videoclip}. That is, we formulate the task
as a video-text retrieval task, except the candidate answers are associated
with each video and the answer selected is the one that is most relevant out of
the four options. We use the same statement templates as above, with the masked
section replaced by each of four multiple choice options, to again reduce the
domain shift from the pre-training data. The model was pre-trained on
HowTo100M~\citep{miech2019howto100m}, a large-scale dataset of instructional
videos. We found that using these statements produced better results than
keeping the question-answer format. Each candidate answer is ranked according
to the InfoNCE contrastive loss~\citep{oord2019infonce}.

\newgeometry{bottom=4cm}
\begin{landscape}
\begin{table}[t]
    \centering
    \caption{Templates for conversion of all question types to statement types
        in STAR. An ``\_'' in a statement indicates the masked answer.  POS and
        syntax tags are as described in the Penn
        Treebank~\citep{marcus1993penntreebank}. Verbs in Sequence\_T1 and
        Sequence\_T2 are post-processed to ensure grammaticality.}
    \label{tab:qs_to_stmts}
    \scriptsize
    \begin{tabular}{lcc}
        \toprule
        Type & Question & Statement \\
        \midrule
        Interaction\_T1 & Which object was [VBD] by the person? & The object [VBD] by the person was \_. \\
        Interaction\_T2 & What did the person do with [NP]? & The person \_ [NP]. \\
        Interaction\_T3 & What did the person do while they were [VBG] the [NP]? & The person \_ while they were [VBG] the [NP]. \\
        Interaction\_T4 & What did the person do while they were [VBG1] [NP1] and [VBG2] [NP2]? & The person \_ while they were [VBG1] [NP1] and [VBG2] [NP2]. \\
        \midrule
        Sequence\_T1 & Which object did the person [VB1] after they [VBP2] [NP]? & The person [VBP1] \_ after they [VBP2] [NP]. \\
        Sequence\_T2 & Which object did the person [VB1] before they [VBP2] [NP]? & The person [VBP1] \_ before they [VBP2] [NP]. \\
        Sequence\_T3 & What happened after the person [VBP] [NP]? & The person \_ after they [VBP] [NP]. \\
        Sequence\_T4 & What happened before the person [VBP] [NP]? & The person \_ before they [VBP] [NP]. \\
        Sequence\_T5 & What did the person do to [NP2] after [VBP1] [NP1]? & The person \_ [NP2] after [VBP1] [NP1]. \\
        Sequence\_T6 & What did the person do to [NP2] before [VBP1] [NP1]? & The person \_ [NP2] before [VBP1] [NP1]. \\
        \midrule
        Prediction\_T1 & What will the person do next? & The person will \_ next. \\
        Prediction\_T2 & What will the person do next with [NP]? & The person will \_ [NP] next. \\
        Prediction\_T3 & Which object would the person [VB] next? & The person would [VB] \_ next. \\
        Prediction\_T4 & Which object would the person [VB2] next after they [VB1] [NP]? & The person would [VB2] \_ next after they [VB1] [NP]. \\
        \midrule
        Feasibility\_T1 & Which other object is possible to be [VBD] by the person? & The other object possible to be [VBD] by the person is \_. \\
        Feasibility\_T2 & What else is the person able to do with [NP]? & The person is also able to \_ with [NP]. \\
        Feasibility\_T3 & Which object is possible to be [VBP] when the person is [PREP] [NP]? & The object possible to be [VBP] when the person is [PREP] [NP] is \_. \\
        Feasibility\_T4 & What is the person able to do when they are [PREP] [NP]? & The person is able to \_ when they are [PREP] [NP]. \\
        Feasibility\_T5 & Which object is the person able to [VB1] after [VBG2] [NP]? & The person is able to [VB1] \_ after [VBG2] NP. \\
        Feasibility\_T6 & What is the person able to do after [VBG] [NP]? & The person is able to \_ after [VBG] [NP]. \\
        \bottomrule
    \end{tabular}
\end{table}
\end{landscape}
\restoregeometry
\normalsize


\section{Zero-Shot Results}
\label{sec:zs_star}

\Cref{tab:zs_star} shows the reported results on STAR for both models. Our
results on Merlot Reserve differ from the authors' reported results slightly;
this may be down to different conversions from questions to statements. The
models are generally comparable to each other, except VideoCLIP appears to be
better at feasibility questions (e.g. ``Which object is possible to be
taken when the person is in front of the table?'') than Merlot Reserve.

\begin{table}[tp] 
    \centering 
	\caption{Results of models (accuracy) on STAR dataset. I, S, P, F stands
	for Interaction, Sequence, Prediction, and Feasibility. We focus on
	sequence questions, which generally require the most temporal reasoning.}
	% For val, 43.01 mean is weighted by num in each group
    \label{tab:zs_star} 
    \begin{tabular}{lccccc} 
        \toprule
        \multicolumn{1}{c}{}        & \multicolumn{4}{c}{Question Types}                & \multicolumn{1}{c}{} \\
                                      \cmidrule(){2-5}
                                    & I           & S        & P          & F           & Mean \\
        \cmidrule(r){1-1}             \cmidrule(){2-5}                                    \cmidrule(l){6-6}
		Chance						& \multicolumn{5}{c}{25.00} \\
        \cmidrule(r){1-1}             \cmidrule(){2-6}
        %Merlot Reserve (val)        & 43.12       & 42.33    & 43.27      & 47.14       & 43.01 (43.97) \\
		Merlot Reserve (val)        & 43.12       & 42.33    & 43.27      & 47.14     & \textbf{43.97} \\
		VideoCLIP (val)             & 39.66       & \textbf{42.86}    & 48.72    & 50.82       & 42.84 \\
        \cmidrule(r){1-1}             \cmidrule(){2-6}
		Merlot Reserve (test)       & 40.51       & \textbf{44.76}    & 43.85    & 39.48       & 42.15 \\  
		VideoCLIP (test)            & 39.77       & 43.60    & 42.60      & 47.13     & \textbf{43.27} \\
        \cmidrule(r){1-1}             \cmidrule(){2-6}
        Merlot Reserve Paper (test) & 44.8        & 42.4     & 38.8       & 36.2        & 40.5 \\
        \bottomrule
    \end{tabular} 
\end{table} 

Having established the baseline results, we go on to test how we can explore
the temporal reasoning ability through a series of data perturbations.


\section{Changing Temporal Indicators}
\label{sec:change_temp}

The first perturbation explores how the models perform when temporal
expressions are changed, so that a question asks for events occurring at
different times in the video to what the question actually asks, while keeping
the `correct' answer the same. For models that are able to reason across time,
we hypothesise that this should result in worse performance because of the
uncertainty caused by the incorrect time ordering in the question. A model that
does not have temporal reasoning ability would have its capabilities limited
lightly to not at all by this perturbation.

We test on the sequence subset of questions, since every sequence question has
a temporal expression~(before/after) in it, and reverse temporal indicators for
each question. For example, the statement ``The person opened \_ \emph{after}
they took the sandwich'' becomes ``The person opened \_ \emph{before} they took
the sandwich'', with the masked answer remaining \emph{the same} for both
statements.

We report results in~\cref{tab:swap_star}. For both Merlot Reserve and
VideoCLIP models, there is very little difference between either test. In fact,
for Merlot Reserve, the accuracy on this subset actually increases slightly
from the correct statements. This suggests that other factors than temporal
reasoning are contributing to the model's performance in this task.

One reason may be the selection of the other multiple choice options. As
described in~\cref{sec:star}, options are generated to provide distractors
based on compositionality, randomness, and frequency, but not for temporality.
So a model may still pick the most likely option based on its ability to
identify the answer using object detection, where the other three options are
not present in the video at all. To counteract this possibility, we set up a
binary-choice experiment, whereby we change the masked answers of statements to
mask the temporal indicator before or after, creating a binary choice, and
replace the previously masked answer with the correct answer from the dataset.
Since we do not have gold labels for the test data, results are only reported
for validation data. As seen in the `Masked Temporal Indicators' column
in~\cref{tab:swap_star}, both models perform only marginally better than chance
in this instance, suggesting that there is no sense of before or after in
either model.

\begin{table}[tp]
    \centering
	\caption{Statement perturbation for Sequence question types on STAR.
	Swapped indicates that a question has had its temporal expression
	(before/after) reversed. Mask Temporal Expressions asks the model to
	predict, given the two actions, whether one happened before or after
	the other, by masking out the temporal expression. Since the test set
	is withheld, we report results for this only on the validation set.}
    \label{tab:swap_star}
    \begin{tabular}{lccc}
        \toprule
        \multicolumn{1}{c}{}     & \multicolumn{3}{c}{Sequence Questions} \\
                                   \cmidrule(){2-4}
        Model          & Correct & Swapped & Mask Temporal Expressions\\
        \cmidrule(r){1-1} \cmidrule(){2-3} \cmidrule(l){4-4}
		Chance				  & \multicolumn{2}{c}{25.00} & 50.00 \\
        \cmidrule(r){1-1} \cmidrule(){2-3} \cmidrule(l){4-4}
        Merlot Reserve (val)  & 42.33   & 42.92 (+0.59)  & 50.86 \\
        VideoCLIP (val)       & 42.86   & 41.91 (-0.95)  & 50.11 \\
        \cmidrule(r){1-1} \cmidrule(){2-3} \cmidrule(l){4-4}
        Merlot Reserve (test) & 44.76   & 41.04 (-3.72)  & {---} \\
        VideoCLIP (test)      & 43.60   & 43.54 (-0.06)  &  {---} \\
        \bottomrule
    \end{tabular}
\end{table}

\section{Randomise Video Frames}

In addition to probing language understanding of temporal expressions, we test
how well the two models encode visual features across frames. Following
\citet{sevilla-lara2021temporal}, who found that performance on action
classification degraded in humans when presented with out-of-order video
frames, we randomise the order of frames presented to the models, and test on
the default created statements. Concretely, for Merlot Reserve this involves
shuffling the order of the 8 frames presented to the model, and aligning the
statement to the new first frame. For VideoCLIP we take the 512-d feature
vectors computed from every 30 frames (see~\cref{sec:videoclip}), and shuffle
based on the time axis. The mean video time is 29.8 seconds, so this involves
shuffling on average 30 feature vectors for each clip. We also experiment with
extracting features from videos that have had all their frames randomly
shuffled. Since Merlot Reserve samples at the frame level anyway, there is no
difference in the two approaches for Merlot Reserve, whereas the computation of
features is affected by a video that has been randomly shuffled for VideoCLIP.
Results are shown in~\cref{tab:shuffle_feats}.

\begin{table}[tp]
    \centering
	\caption{Probing video perturbations. Shuffled video features accuracy on
	STAR test set. VideoCLIP results include both shuffling frame-level features,
	and computing features on a shuffled video. Probing video features results
	in a slight decrease in performance, with a large decrease for shuffled video.}
    \label{tab:shuffle_feats}
    \begin{tabular}{lccccc}
        \toprule
        \multicolumn{1}{c}{}      & \multicolumn{4}{c}{Question Types}  & \multicolumn{1}{c}{} \\
        \cmidrule(){2-5}
        Model                     & I & S & P & F & Mean \\
        \cmidrule(r){1-1} \cmidrule(){2-5} \cmidrule(l){6-6}
        %Merlot Reserve            & 42.08       & 42.58    & 47.28      & 49.2        & 43.28 (45.29) \\
        %Original Merlot Reserve   & 43.12       & 42.33    & 43.27      & 47.14       & 43.01 (43.97) \\
        %\cmidrule(r){1-1} \cmidrule(){2-5} \cmidrule(l){6-6}
        %VideoCLIP (shuffle video) & 34.61       & 36.31    & 36.70      & 40.82       & 36.08 \\
        %VideoCLIP (shuffle frame) & 39.99       & 43.06    & 46.47      & 49.59       & 43.39 \\
        %Original VideoCLIP        & 39.66       & 42.86    & 48.72      & 50.82       & 42.84 \\
		%\midrule
%		On test set & & & & &\\
        Original Merlot Reserve   & 40.51       & 44.76    & 43.85      & 39.48       & 42.15 \\  
		Merlot Reserve			  & 38.88		& 40.61	   & 42.32		& 39.48		  & 40.32 \\
        \cmidrule(r){1-1} \cmidrule(){2-5} \cmidrule(l){6-6}
        Original VideoCLIP        & 39.77       & 43.60    & 42.60      & 47.13       & 43.27 \\
		VideoCLIP (shuffle frame) & 39.27		& 43.31	   & 42.04		& 46.09		  & 42.68 \\
		VideoCLIP (shuffle video) & 33.67       & 35.91    & 36.17      & 34.96       & 35.18 \\
        \bottomrule
    \end{tabular}
\end{table}

We would expect that models with shuffled video features perform worse than the
correct video features. With the exception of the shuffled video for VideoCLIP,
there is little difference in performance between the setups. Computing
features from randomly shuffled videos results in significantly worse
performance due to the weaker features that can be learned from a jumbled-up
video, where features are extracted in the temporal dimension. Shuffling
well-made features, or images in the case of Merlot Reserve, at the frame level
is reasonably robust to these effects, although there is a small drop in
performance.

\section{Summary}
\label{sec:probe_summary}

In this chapter we have looked at the existing capabilities for temporal
reasoning of two state of the art \acrlongpl{vidlm}, Merlot Reserve and
VideoCLIP. We found that their competitive performance on a \acrshort{vidqa}
dataset with questions that should require temporal understanding was not
because of the learned abilities of the models through a targeted probing
setup. By perturbations of the evaluation data, for both language and video, we
have shown that the models are not sensitive to misdirecting inputs, and are
unable to identify (with random chance) the correct ordering of two actions
when asked to determine whether one action occurs before or after another. In
the next chapter we propose a method for learning fine-grained temporal
understanding in the Merlot Reserve model by additional training on targeted
data.


% \section{VideoCLIP STAR results}
% \label{sec:videoclip}
% 
% Zero-shot, statements as in Merlot Reserve
% 
% Validation set: 42.84 \\
% Validation sequence only: 42.86 \\
% Validation sequence only swap before/after in statement: 41.91 \\
% Validation with shuffled frames (all): 36.08 \\
% Validation with shuffled frames (seq only): 36.31 \\
% 
% Use before/after as masked answers: 50.11
% 
% Shuffled frames are from features extracted every 30 frames, so
% generally more frames are chosen than Merlot Reserve (8), as
% average video time is 29.8 seconds.
% Bigger drop-off in performance, but still much better than random
% chance.

%! TeX root = ../charles/en/thesis.tex
\chapter{Experiment Setup}
\label{chap:setup}

We continue training the Merlot Reserve model on videos from the Charades
dataset. We call this process post-pre-training, rather than finetuning, since
the objective function is the same as in the pre-training stage, but we train
on a smaller dataset that is designed to improve the temporal reasoning of the
Merlot Reserve model before being evaluated on downstream tasks. This chapter
explains the creation process of this small finetuning dataset, as well as the
post-pre-training process.

\section{Dataset Creation}
\label{sec:data}


For each video, we create new relation pairs based on annotated actions. Each
relation is based on Allen's Interval Algebra~\citep{allen1983interval}, shown
in Table \ref{tab:allen_interval}. 

\begin{table}[tp]
	\centering
	\caption{The Thirteen Possible Relationships. All relation types except for
		\textit{equal} have a corresponding inverse relation type. Modified
		from~\citet{allen1983interval}}
	\label{tab:allen_interval}
	\begin{tabular}{ll}
	Relation & Pictoral Example \\
	\hline
	\texttt{X} \textit{before} \texttt{Y} & \texttt{XXX YYY} \\
	\texttt{X} \textit{equal} \texttt{Y} & \texttt{XXX} \\
					   & \texttt{YYY} \\
	\texttt{X} \textit{meets} \texttt{Y} & \texttt{XXXYYY} \\
	\texttt{X} \textit{overlaps} \texttt{Y} & \texttt{XXX} \\
						& \texttt{ YYY} \\
	\texttt{X} \textit{during} \texttt{Y} & \texttt{ XXX} \\
						& \texttt{YYYYYY} \\
	\texttt{X} \textit{starts} \texttt{Y} & \texttt{XXX} \\
						& \texttt{YYYYY} \\
	\texttt{X} \textit{finishes} \texttt{Y} & \texttt{\space\space XXX} \\
							& \texttt{YYYYY} \\
	\end{tabular}
\end{table}

\subsection{Aligning Frames}
\label{ssec:frames}

For each video we have all relations $(X, Y)$ between pairs of actions $X, Y$.
We then attempt to include frames related to as many relations as possible,
with the requirement that relations cannot overlap with one another in time.
Where there is an overlap, the precedence order of relations is as follows: \textit{meets,
overlaps, starts, finishes, during, equals, precedes}. %TODO: why?

Frames are selected based on $(X_{start}, X_{end}), (Y_{start}, Y_{end})$ and
the specific relation type. If a relation requires frames that intersect an
already selected frame, the relation is discarded. This is to keep a strict
relationship between actions and time relations. Once all relations have been
processed, any remaining frames up to 8 (the number of frames used in Merlot
Reserve) are selected uniformly at the beginning or end, depending on the time
before and after any relations. %Maybe more detail on exactly what is done here?

Each frame associated with a relation is annotated with the associated actions
$X$ and $Y$, along with a temporal indicator based on the specific relation
type. 

%For example, for a 30 second video, one relation may select two frames
%at 11 and 17 seconds, based on the times of the relation's actions. If a second
%relation requires frames at 15 and 19 seconds, this relation would be scrapped,
%since the alignment of the text description to their respective frames becomes
%impossible.

\subsection{Positive Labels}
\label{ssec:pos_labels}

Positive labels are created depending on the relation type, and consist of two
actions $X$ and $Y$, and temporal markers indicating the relation between the
two actions. The labels are split across multiple frames as appropriate
depending on the relation type. For example, a \textit{finishes} relation type
is split across two frames, with the first frame being $Y$ + \{then, before\},
and the second $X$ + \{while, at the same time as\} + $Y$. 
%TODO: probably more detail, maybe a chart here.

\subsection{Contrastive Span Objective}
\label{ssec:contr_span}

We use a slightly modified contrastive span objective, as
in~\citet{zellers2022mreserve}. The difference between our implementation and
the original comes with the masking strategy. We restrict possibly masked spans
to spans which contain a temporal word. Since we continue training on the
pretrained Merlot Reserve model, we do not require further training of the
general span objective, but focus explicitly on the learning of temporal
reasoning between relations. We do this by using additional hard negatives
focused on temporal words, along with negatives chosen randomly from the batch.
The hard negatives act as a close match to the positive option in the
contrastive setup, but are specifically wrong in the temporal dimension. This
focuses the model on learning how to reason across time.

\subsection{Creating Negative Spans}
\label{ssec:neg_labels}

We create a list of negative spans for each relation type. Negative spans are
spans that match the corresponding positive span, except for temporal markers
in the span. The temporal marker is changed to an alternative temporal marker
that does not reflect the order of events as determined by the relation. For
example, the relation type \textit{precedes} is mapped to a set of negative
temporal markers \textit{``after", ``while", ``at the same time as"}. Each
negative span consists of the positive span with the positive temporal marker
substituted for a negative temporal marker. These are then fed to the model
as additional hard negatives for the contrastive span objective.

\begin{figure}[tp]
	\centering
	\includegraphics[width=0.8\textwidth]{hard_negatives}
	\caption{Example of annotation with hard negatives for temporal words}
	\label{fig:hard_neg}
\end{figure}


\section{Merlot Reserve Post-Pre-Training}
\label{sec:training}

Using this dataset, we continue training Merlot Reserve with a similar
pretraining objective as described in the original paper. We emphasise the
relevant steps here, but see~\citet{zellers2022mreserve} for full details.

%! TeX root = ../charles/en/thesis.tex
\chapter{Temporal Ability of Vision and Language Models}

%%! TeX root = ../charles/en/thesis.tex
\chapter{Temporal Ability of Vision and Language Models}
\label{chap:discussion}

Discussion of results, thoughts on contrastive pretraining, other insights/ideas
as they come up.

%Different models have different approaches to modelling temporal awareness for
%videos, even within the contrastive pretraining approach. Merlot
%Reserve~\citep{zellers2022mreserve} relies on an alignment between subtitles,
%audio, and video frames to keep consistent temporal awareness with the
%contrastive span objective, but its architecture lacks a specific module for
%temporal reasoning. Similarly, VideoCLIP~\citep{xu2021videoclip} relies on
%temporally overlapping video-text pairs to train its contrastive objective
%function with an otherwise simple architecture.
%
%In contrast, other models explicitly include temporal modules in the
%architecture, either through learnt temporal positional
%encodings~\citep{alayrac2022flamingo}, 3D linear projections of video
%features~\citep{luo2022clip4clip}, temporal convolutions~\citep{???},
%cross-attention between frames~\citep{???}, or combinations
%thereof~\citep{lin2022evl,bain2021frozen}. \citet{lin2022evl} find that the effect of temporal
%information varies greatly depending on the dataset, and analysis on
%Something-Something-V2, which relies more heavily on temporal information,
%shows that including a single temporal module improves accuracy by over 10\%,
%while combining sources of temporal information provides marginal extra
%performance gain.
%
%
%Faithful temporal reasoning may be made possible by chaining reasoning steps in
%language models, e.g.~\citet{creswell2022faithful}
%
%How to connect symbolic models of temporal reasoning with current work in
%developing language and vision models is difficult.


\chapter*{Conclusion}
\addcontentsline{toc}{chapter}{Conclusion}
%! TeX root = ../charles/en/thesis.tex
\label{chap:conclusion}



%%% Bibliography
\include{bibliography}

%%% Figures used in the thesis (consider if this is needed)
\listoffigures

%%% Tables used in the thesis (consider if this is needed)
%%% In mathematical theses, it could be better to move the list of tables to the beginning of the thesis.
\listoftables

%%% Abbreviations used in the thesis, if any, including their explanation
%%% In mathematical theses, it could be better to move the list of abbreviations to the beginning of the thesis.
%\chapwithtoc{List of Abbreviations}

\printglossary[type=\acronymtype, title=List of Abbreviations, toctitle=List of Abbreviations]

%%% Attachments to the master thesis, if any. Each attachment must be
%%% referred to at least once from the text of the thesis. Attachments
%%% are numbered.
%%%
%%% The printed version should preferably contain attachments, which can be
%%% read (additional tables and charts, supplementary text, examples of
%%% program output, etc.). The electronic version is more suited for attachments
%%% which will likely be used in an electronic form rather than read (program
%%% source code, data files, interactive charts, etc.). Electronic attachments
%%% should be uploaded to SIS and optionally also included in the thesis on a~CD/DVD.
%%% Allowed file formats are specified in provision of the rector no. 72/2017.
%\appendix
%\chapter{Attachments}

\section{First Attachment}


\openright
\end{document}
