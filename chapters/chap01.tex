%! TeX root = ../charles/en/thesis.tex

\chapter{Background}
\label{chap:bg}

In this chapter, we briefly cover progress in obtaining useful representations
of language (Section \ref{sec:lm}), images (Section \ref{sec:imrec}), and
efforts to combine these two modalities (Section \ref{sec:vlm}). We then
discuss the extension of vision and language models to videos (Section
\ref{sec:vidlm}), which introduces the extra complexity of reasoning across
sequences of images, and optionally adding a further modality, audio. 
%We finish the chapter with an introduction to and motivation for studying the
%task of video question answering (Section \ref{sec:vidqa}).

\section{Language Modeling}
\label{sec:lm}

\subsection{Recurrent Neural Networks}
\label{ssec:rnn}

Vanilla RNN, LSTM, GRU, attention
	
\subsection{Transformer}
\label{ssec:transformer}

\cite{vaswani2017attention} introduced the Transformer architecture for sequence tasks, replacing the recurrent nature of the RNN and its variants with multi-head self-attention

\section{Image Recognition}
\label{sec:imrec}

Object detection, classification

\subsection{Convolutional Neural Networks}
\label{ssec:cnn}


\subsection{Vision Transformer}
\label{ssec:vit}

\cite{dosovitskiy2021vit} introduced the Vision Transformer (ViT), which takes the impressive performance of the Transformer architecture on sequence tasks and applies it to image tasks. The authors represent an image as a sequence of patches of an image, with an extra patch embedding added alongside the positional embedding of the Transformer \cite{vaswani2017attention} to maintain the 2-dimensional information of an image when projected into a linear sequence. 

\section{Vision and Language Models}
\label{sec:vlm}

\subsection{CLIP}
\label{ssec:clip}

\subsection{Visual Question Answering}
\label{ssec:vqa}

\section{Video Language Models}
\label{sec:vidlm}

%\section{Video Question Answering}
%\label{sec:vidqa}
%
%\subsection{STAR Dataset}
%\label{ssec:star}
%
%We primarily focus on this due to its focus on sequential questions that evaluate model performance on temporal reasoning
%
%\subsection{Merlot Reserve}
%\label{ssec:mreserve}
%
%We examine a specific video language model, Merlot Reserve \citep{zellers2022mreserve}, that performs strongly on STAR.
