
%\Universita{University of Trento}
%\Dipartimento{Center for Mind/Brain Sciences}
\CorsoDiLaurea{Master’s Degree in Cognitive Science}
\AnnoAccademico{Academic Year\\2022--2023}
\Titolo{TEMPORAL REASONING IN VISION AND LANGUAGE MODELS}
\Relatore{Prof. Raffaella \textsc{Bernardi}}
\RelatoreLabel{Supervisor}
\Correlatore{Prof. Paolo \textsc{Rota}}
\CorrelatoreLabel{Co-Supervisor}
\CandidatoLabel{Student:}

\Candidato{Andrew \textsc{McIsaac}} 
%\Matricola{236671}
%\DataEsame{\today}
%\Logo{Impaginazione/logo_rosso.png}
\Logo{trento_header}
\LogoWidth{10cm} %optional, default: 3cm
\LogoPosition{top}
\LogoSfondo{Impaginazione/logo_bn.png}
\opacitaSfondo{0.00}

\begin{titlepage}
    \newgeometry{left=3cm, right=3cm, bottom=2cm, top =3cm} 
    \pagestyle{empty}
    \makefrontpage
    \restoregeometry
\end{titlepage}

\frontmatter
% dedica
%~ \newpage
%\null\vspace{\stretch{1}}
%\begin{flushright}
%    \textit{Dedica}
%\end{flushright}
%\vspace{\stretch{2}}\null

%ringraziamenti
\chapter*{Acknowledgments}

Thank you first, to my supervisors, Prof. Raffaella Bernardi and Prof. Paolo
Rota, for introducing this engaging topic to me, and helping to guide me
through this thesis-writing process, and for providing me access to compute
clusters, without which I would not have been able to do this work. Thank you
also to my co-supervisor at Charles University, Doc. RNDR. Ondřej Bojar, and to
all my professors who have taught me throughout this program. I would also like
to thank the Erasmus Mundus Language and Communication Technologies program for
generously providing me with a scholarship during my Master's, and for giving
me the opportunity to travel and spend an extended period of time in two
wonderful countries. 

But the places wouldn't be anything without the people you meet, so thank you
to the friends I have made along the way, for your support and kindness
throughout. Finally, thank you to my family, who have supported me from afar
over the last two years.


\begin{abstract}
	Are vision and language models able to reason across time? We evaluate the
	performance of vision and language models (VLMs) on the task of video
	question answering, with a particular focus on their temporal reasoning
	abilities. We probe the STAR video QA dataset on two VLMs with data
	perturbation methods of text and video inputs, and find that models are
	generally unable to identify the meaning of before and after in sequential
	questions. We then ask how a model can effectively learn these temporal
	relations, and design a new dataset drawn from videos and annotations from
	the Charades dataset. We create annotations that include targeted hard
	negative examples for the contrastive loss objective of one VLM, Merlot
	Reserve, such that the model must adapt to learn temporal relations. We
	further explore how to model fine-grained temporal relationships, and
	evaluate the benefits. We find that our approach shows promising signs of
	improvement on tasks that require temporal understanding, although it gains
	little sensitivity to temporal relations when probed.
\end{abstract}

\tableofcontents

%\printglossaries
%\addcontentsline{toc}{chapter}{Glossary}

%\printnomenclature
%\addcontentsline{toc}{chapter}{Nomenclature list}

\mainmatter
\chapter*{Introduction}
\markboth{\MakeUppercase{Introduction}}{}
\addcontentsline{toc}{chapter}{Introduction}
%! TeX root = ../charles/en/thesis.tex

%A bit on how vision and language models perform well on many benchmarks. Then
%talk about videos. How vision and language models have been extended to video
%tasks via extra training on paired video-caption datasets, frozen LMs.
%
%Introduce the problem. Main question: do they learn temporal reasoning? Are
%models able to learn the difference between action X happening before action Y,
%versus action X happening after action Y.
%
%Define vision and language model to be the class of models of which a subset is
%video and language models.
%
%What do we do? - Discover that the answer is no, at least for Merlot Reserve.
%How? Masking temporal indicator in STAR, no idea if it is before/after.
%\XXX{Ondrej: A question for you or Raffa: Discovering that the answer is no is actually fortunate, the easier option. What is unclear is how would we convince ourselves that the answer is yes. What methodology would we use?}
%
%Perhaps this should be explored for some other models as well (e.g. ClipBERT
%\cite{lei2021clipbert}, Flamingo \cite{alayrac2022flamingo}, Frozen CLIP
%\cite{lin2022evl}, VidIL \cite{wang2022vidil}, Socratic Models
%\cite{zeng2023socratic}, VideoCLIP \cite{xu2021videoclip})
%\XXX{Ondrej: yes! But primarily focus on the promised Merlot Reserve. If you can easily (one push of a button) do more, do as many as you can.}
%
%So, mask temporal indicators in scripts, add negatives for cues.
%\XXX{Ondrej: without knowing the details, I can't imagine any such cues yet. Maybe provide already an example in the introduction.}
%E.g. before -\textgreater [after, at the same time, while]
%Or possibly masking actions, how to generate negatives for this isn't clear.
%Would need multiple masks potentially, quite complicated.
%
%Train with either YT-Temporal-1B or Charades. Does it improve performance on
%answering questions that require temporal information? If Charades, probably
%need to do other datasets as well, e.g. NextQA, Epic Kitchens.
%
%Hopefully the answer is that it performs better.
%\XXX{For ``better'' you need a continuous measure (which you will certainly and easily have), and an improvement in this measure. Yet my high-level methodological question remains: What score in this measure would the model need in order to say ``the model does learn temporal reasoning''.}

Research in \acrfullpl{vlm} has bloomed over recent years. With larger and
larger datasets and models, particularly based on the
Transformer architecture~\citep{vaswani2017attention}, the performance and
capabilities of \acrshortpl{vlm} have increased on common multimodal tasks such
as visual question answering, image captioning, visual dialogue generation and
image-text retrieval
\citep{alayrac2022flamingo,li2022blip,li2023blip2,radford2021clip}.

\Acrfullpl{vidlm} are \acrshortpl{vlm} which are capable of modelling video.
This provides the additional challenge of modelling long sequences of frames,
and reasoning temporally across these images. Models must be able to recognise
how a scene changes over time not only with respect to objects and relations
between them, but they must also model the causal link between actions and
events. For tasks such as video question answering, where a model is given a
video and a question, and must pick the correct answer out of a number of
multiple choice options, to answer questions such as ``What did the man do
after opening the door?'', or ``Why was the toddler crying at the end of the
video?'', it must be able to relate potentially distant events to one another
and reason about them.  Even if a model is able to select the correct option,
how do we know that it has applied the correct reasoning steps required to make
its prediction?  \citet{lei2023revealing} and \citet{buch2022revisiting} show
that models trained with just a single frame can match or outperform the state
of the art on multiple video and language tasks. This suggests that existing
evaluation datasets have a ``static appearance bias''~\citep{lei2023revealing}
without challenging enough questions or options that would require event-level
understanding to distinguish between them, and potentially that pre-training
datasets and objectives are not incentivised to learn temporal %\XXX{the following should be citep, not cite}
information~\citep{momeni2023verbs}. We explore whether current downstream
datasets are able to show that temporal reasoning has been learned, for
instances that should require temporal reasoning. \Acrshortpl{vidlm} are often
trained using contrastive learning, where the objective is to correctly match
video-text pairs to each other, while repelling non-matching pairs in the joint
embedding space. 

In this thesis, we aim to understand the abilities of \acrshortpl{vidlm} to
reason across time, and to propose a method for instilling a fine-grained
temporal reasoning ability in such models. We design perturbation experiments
to look at the temporal reasoning abilities of multiple \acrlongpl{vidlm}
trained with a contrastive objective function. Following work that questions
the ability of contrastive learning to pay attention to order structure in
\acrshortpl{vlm}~\citep{yuksekgonul2023when}, we ask a similar question of
their ability to reason across time. Can \acrshortpl{vidlm} learn a grounded
representation of temporal relations, especially ``before'' and ``after''?
Using questions which require sequential information from the STAR %\XXX{the following should be citep, not cite; and similarly in more places; you should check this everywhere}
dataset~\citep{wu2021star}, a \acrfull{vidqa} dataset which tests situated
reasoning questions in real-world videos, we find that these models do not
learn to distinguish between actions occurring before or actions occurring
after one another.

Following this finding, we ask how a model can be incentivised to learn to
encode these grounded representations. We propose a method for learning
temporal reasoning abilities, using targeted hard negatives in the contrastive
objective to improve the model's understanding of temporal relations. We use
videos from the Charades dataset~\citep{sigurdsson2016charades}, which has
annotated events and their corresponding timespans, to create annotations with
a temporal relation connecting a pair of actions, and hard negatives which
modify the annotation in the temporal dimension only. For example, in a video
that has the action annotations ``someone is dressing'' and ``taking a cup from
somewhere'', with the first action occurring before the second, we generate the
temporally-aware label ``someone is dressing before taking a cup from
somewhere''. We then create hard negatives that modify the label in the
temporal dimension (e.g. ``someone is dressing \textit{after} taking a cup from
somewhere''), which is added to the batch of video-text pairs as a non-matching
pair. 

We evaluate our approach on multiple \acrshort{vidqa} datasets to test
different types of temporal understanding and the generalisability of our
approach. We also explore the effect of using a wider range of temporal
relations than just before and after to model more fine-grained relations. We
use relations from Allen's Interval Algebra~\citep{allen1983interval}, a
calculus for reasoning about temporal intervals, which defines a range of
temporal relation types that capture possible relations between actions.  We
find that using more fine-grained relations improves performance compared to
both using just before and after, and on downstream tasks, although we find
limited evidence that our models become more robust to temporal reasoning
probing tests.

The rest of this thesis is organised as follows:

\begin{itemize}
	\item \Cref{chap:bg} goes into the background of language models, image
		recognition models, and \acrfullpl{vlm} which combine the two
		modalities. We discuss one popular method for training, \Acrfull{clip},
		and one key downstream task, \acrlong{vqa}. We finish the chapter with
		a broad overview of video language models and an overview of the
		temporal reasoning literature in video and in language.
	\item \Cref{chap:rel} explores related work on \acrlongpl{vidlm},
		with a particular focus on work that explores the impact of contrastive
		pre-training. We highlight previous work that has explored
		temporal reasoning in \acrlongpl{vidlm}.
	\item In \cref{chap:dataset} we look at the main datasets used, STAR and
		NExT-QA, as well as the particular pre-trained models that we work on.
	\item In \cref{chap:probe} we test the current temporal reasoning abilities
		of multiple video language models on the STAR dataset.
	\item \Cref{chap:setup} details experiments that show how current
		models perform on temporal reasoning tasks, and describes our approach
		to generating additional hard negatives focussing on temporal words for
		contrastive training.
	\item \cref{chap:results} shows performance of our model on STAR and NExT-QA.
		We evaluate different design decisions in our dataset, and compare our
		approach to related work.
%	\item \cref{chap:discussion} discusses the use of contrastive
%		pre-training methods in video and language models, and how our method
%		affects performance of temporal reasoning systems.
	\item Finally, the \nameref{chap:conclusion} summarises our findings and
		discusses the use of contrastive pre-training methods in
		\acrlongpl{vidlm} for temporal reasoning.
\end{itemize}

